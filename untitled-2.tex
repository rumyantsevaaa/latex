\documentclass{article}
\usepackage[left=0.5cm,right=0.5cm,top=2cm,bottom=2cm,nohead,nofoot]
{geometry}
\usepackage[utf8]{inputenc}
\usepackage[russian]{babel}


\title{1}
\author{Румянцева Елена}
\date{February 2018}

\begin{document}
\begin{center}
  \par\bigskip  \Large\bfseries Семнадцать уравнений,которые изменили мир от Яна Стюарта
\end{center}
\begin{tabular}[t]{1 1 1}

\par\bigskip {\normalsize {\bf 1. Теорема Пифагора} &  $a^2+b^2=c^2$  & {\bf Пифагор, 530 г.до н.э}\\
\par\bigskip {\bf 2. Логарифмы} &  $\log{xy}&\\
\par\bigskip {\bf 3. Исчисления} & $\frac{df}{dt}=\lim\limits_{x \to 0}\frac{f(t+h)-f(t)}h$ & {\bf Исаак Ньютон, 1668}\\
\par\bigskip {\bf 4. Закон всемирного тяготения} & $F=G\frac{m_1m_2}{~r^2}$ & {\bf Исаак Ньютон, 1687}\\
\par\bigskip {\bf 5. Комплексное число} & $i^2=-1$  & {\bf Л.Эйлер, 1750}\\
\par\bigskip {\bf 6. Эйлерова характеристика} & $V -E+F=2$   & {\bf Л.Эйлер, 1751}\\
\par\bigskip {\bf 7. Нормальное распределение}  & $Ф(x)=\frac{1}{\sqrt{2\pi\rho}}exp(\mu it-\frac{\sigma^2t^2}2)$ & {\bf К.Ф.Гаусс, 1810}\\
\par\bigskip {\bf 8. Волновое уравнние} &                $\frac{\partial^2u}{\partial^2t}=c^2\frac{\partial^2u}{\partial^2x}$ & {\bf Л.Д`Аламблер , 1746}\\
\par\bigskip {\bf 9. Преобразование Фурье}  &  $f(\omega)=\int_\infty^\infty f(x)e^(^-^2^\pi^i^x^\omega^) dx$ & {\bf Ж.Фурье, 1822}\\
\par\bigskip {\bf 10. Уравнение Навье-Стокса} & $\rho(\frac{\partial v}{\partial t}+v\nabla v)=-\nabla p+\nabla T+f$  & {\bf А.Навье,Дж.Стокс, 1845}\\
\par\bigskip {\bf 11. Уравнения Максвелла} &
$\nabla E=\frac{\rho}{~\epsilon_o}$\hspace{1cm}$\nabla H=0 & {\bf Дж.К.Максвелл, 1865}\\\par\bigskip &\nabla\times E=-\frac{1}{e}\frac{\partial H}{\partial t}\hspace{1cm}\nabla\times H=-\frac{1}{e}\frac{\partial E}{\partial t} & \\
\par\bigskip {\bf 12. Второй закон термодинамики} &  $dS \ge 0$$  & {\bf Л.Больцман, 1874}\\
\par\bigskip {\bf 13. Теория относительности} & $E=mc^2$ & {\bf А.Эйнштейн, 1905 }\\
\par\bigskip {\bf 14. Уравнение Шрёдингера} & $ih\frac{\partial}{\partial t}\Psi=H\Psi$ & {\bf Э.Шрёдингер, 1927 }\\
\par\bigskip {\bf 15.Теория информации } & $H=-\sum p(x)\log{p(x)}$ & {\bf К.Шеннон, 1949 }\\
\par\bigskip {\bf 16.Теория хаоса}  & $~x_t_+_1=k~x_t(1-~x_t)$  & {\bf Р.Мэй, 1975}\\
\par\bigskip {\bf 17.Модель Блэка-Шоулза} & $\frac{1}2\sigma^2S^2\frac{\partial^2V}{\partial S^2}+rS\frac{\partial V}{\partial S}+\frac{\partial V}{\partial t}-rV=0$  & {\bf Ф.Блэк и М.Шоулз, 1990} }
\end{tabular}


\end{document}