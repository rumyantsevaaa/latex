\documentclass{article}
\usepackage[utf8]{inputenc}
\usepackage[russian]{babel}

\title{1}
\author{Румянцева Елена}
\date{February 2018}

\begin{document}
\begin{center}
{\bf Монотонная последовательность}
\end{center}
Последовательность { ~$x_n$} называют { возрастающей (неубывающей)},если для любого n \in N выполняется неравенство
\begin{center}
~$x_n_+_1$ $>~$x_n$   (1)
\end{center}
Аналогично последовательность {$\it ~$x_n$} называют убывающей (невозрастающей),если для любого  n \in N справедливо неравенство 
\begin{center}
~$x_n_+_1$<~$x_n$   (2) 
\end{center}
Если неравенсто (1) можно записать в виде {~$x_n_+_1$>~$x_n$} ,а неравенство (2) - в виде ~$x_n_+_1$<~$x_n$ ,то последовательность {~$x_n$} называют соответсвенно строго и возрастающей и строго убывающей.
Точную верхнюю (нижнюю) грань множества значений последовательности и обозначают соответственно \sup{~$x_n$} и \inf{~$x_n$}.
\end{document}
